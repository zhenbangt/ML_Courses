\clearpage
\item \subquestionpoints{5}
Recall that in GDA we model the joint distribution of $(x, y)$ by the following
equations:
%
\begin{eqnarray*}
	p(y) &=& \begin{cases}
	\phi & \mbox{if~} y = 1 \\
	1 - \phi & \mbox{if~} y = 0 \end{cases} \\
	p(x | y=0) &=& \frac{1}{(2\pi)^{n/2} |\Sigma|^{1/2}}
		\exp\left(-\frac{1}{2}(x-\mu_{0})^T \Sigma^{-1} (x-\mu_{0})\right) \\
	p(x | y=1) &=& \frac{1}{(2\pi)^{n/2} |\Sigma|^{1/2}}
		\exp\left(-\frac{1}{2}(x-\mu_1)^T \Sigma^{-1} (x-\mu_1) \right),
\end{eqnarray*}
%
where $\phi$, $\mu_0$, $\mu_1$, and $\Sigma$ are the parameters of our model.

Suppose we have already fit $\phi$, $\mu_0$, $\mu_1$, and $\Sigma$, and now
want to predict $y$ given a new point $x$. To show that GDA results in a
classifier that has a linear decision boundary, show the posterior distribution
can be written as
%
\begin{equation*}
	p(y = 1\mid x; \phi, \mu_0, \mu_1, \Sigma)
	= \frac{1}{1 + \exp(-(\theta^T x + \theta_0))},
\end{equation*}
%
where $\theta\in\Re^n$ and $\theta_{0}\in\Re$ are appropriate functions of
$\phi$, $\Sigma$, $\mu_0$, and $\mu_1$.

\ifnum\solutions=1{
  \begin{answer}

$$
\begin{aligned}
p(y=1|x; \phi, \mu_0, \mu_1, \Sigma) &= \frac{p(x|y=1; \phi, \mu_0, \mu_1, \Sigma) p(y=1)}{\sum_j  p(x|y=j; \phi, \mu_0, \mu_1, \Sigma) p(y=j)}\\
&= \frac{\exp\left(-\frac{1}{2}(x-\mu_{1})^T \Sigma^{-1} (x-\mu_{1})\right)\ \phi}{ \exp\left(-\frac{1}{2}(x-\mu_{0})^T \Sigma^{-1} (x-\mu_{0})\right)\phi + \exp\left(-\frac{1}{2}(x-\mu_{0})^T \Sigma^{-1} (x-\mu_{0})\right)(1-\phi)} \\
\mbox{divide by top}&= \frac{1}{1 + \exp(-\frac{1}{2}\big[(x-\mu_{0})^T \Sigma^{-1} (x-\mu_{0})-(x-\mu_{1})^T \Sigma^{-1} (x-\mu_{1}) \big])\frac{1-\phi}{\phi}} \\
&= \frac{1}{1 + \exp(-\frac{1}{2}\big[(x-\mu_{0})^T \Sigma^{-1} (x-\mu_{0})-(x-\mu_{1})^T \Sigma^{-1} (x-\mu_{1}) \big]+ln(\frac{1-\phi}{\phi}))} \\
&= \frac{1}{1 + \exp(-\frac{1}{2}\big[ (\mu_1-\mu_0)^T \Sigma^{-1} (\mu_1-\mu_0) + 2 (\mu_1-\mu_0)^T \Sigma^{-1} x \big]+ln(\frac{1-\phi}{\phi}))} 
\end{aligned}
$$
Therefore, $\theta_0 =-\frac{1}{2}\mu_1^T \Sigma^{-1} \mu_1 + \mu_0^T \Sigma^{-1} \mu_0 - ln(\frac{1-\phi}{\phi})$; $\theta = (\mu_1-\mu_0)^T \Sigma^{-1}  $
\end{answer}

}\fi
