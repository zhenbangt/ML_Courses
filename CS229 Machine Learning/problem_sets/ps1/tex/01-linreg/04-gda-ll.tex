\clearpage
\item \subquestionpoints{7} For this part of the problem only, you may
  assume $n$ (the dimension of $x$) is 1, so that $\Sigma = [\sigma^2]$ is
  just a real number, and likewise the determinant of $\Sigma$ is given by
  $|\Sigma| = \sigma^2$.  Given the dataset, we claim that the maximum
  likelihood estimates of the parameters are given by
  \begin{eqnarray*}
    \phi &=& \frac{1}{m} \sum_{i=1}^m 1\{y^{(i)} = 1\} \\
\mu_{0} &=& \frac{\sum_{i=1}^m 1\{y^{(i)} = {0}\} x^{(i)}}{\sum_{i=1}^m
1\{y^{(i)} = {0}\}} \\
\mu_1 &=& \frac{\sum_{i=1}^m 1\{y^{(i)} = 1\} x^{(i)}}{\sum_{i=1}^m 1\{y^{(i)}
= 1\}} \\
\Sigma &=& \frac{1}{m} \sum_{i=1}^m (x^{(i)} - \mu_{y^{(i)}}) (x^{(i)} -
\mu_{y^{(i)}})^T
  \end{eqnarray*}
  The log-likelihood of the data is
  \begin{eqnarray*}
\ell(\phi, \mu_{0}, \mu_1, \Sigma) &=& \log \prod_{i=1}^m p(x^{(i)} , y^{(i)};
\phi, \mu_{0}, \mu_1, \Sigma) \\
&=& \log \prod_{i=1}^m p(x^{(i)} | y^{(i)}; \mu_{0}, \mu_1, \Sigma) p(y^{(i)};
\phi).
  \end{eqnarray*}
By maximizing $\ell$ with respect to the four parameters,
prove that the maximum likelihood estimates of $\phi$, $\mu_{0}, \mu_1$, and
$\Sigma$ are indeed as given in the formulas above.  (You may assume that there
is at least one positive and one negative example, so that the denominators in
the definitions of $\mu_{0}$ and $\mu_1$ above are non-zero.)

\ifnum\solutions=1 {
  \begin{answer}
We have
    $$
    \begin{aligned}
l(\phi, \mu_0, \mu_1, \Sigma) &= \sum_{i=1}^m(\log p(x^{(i)}|y^{(i)}; \mu_0, \mu_1, \Sigma) + \log p(y^{(i)};\phi))\\
&= \sum_{i=1}^m [1\{y^{(i)} = 0\} \log \frac{1}{\sqrt{2\pi}\sigma}\exp(-\frac{(x^{(i)} - \mu_0)^2}{2\sigma^2}) + 1\{y^{(i)} = 1\} \log \frac{1}{\sqrt{2\pi}\sigma}\exp(-\frac{(x^{(i)} - \mu_1)^2}{2\sigma^2}) + y^{(i)} \log \phi + (1 - y^{(i)}\log (1- \phi))]
\end{aligned}
$$
And we set derivatives to zero:

    \[
\begin{aligned}
\frac{\partial l}{\partial \phi} = y^{(i)}\frac{1}{\phi} + (1 - y^{(i)})\frac{1}{1- \phi} = 0\\
\frac{\partial l}{\partial \mu_0} = - \sum_{i=1}^m1\{y^{(i)} = 0\} \frac{x^{(i)} - \mu_0}{\sigma^2} = 0\\
\frac{\partial l}{\partial \mu_1} = - \sum_{i=1}^m1\{y^{(i)} = 1\} \frac{x^{(i)} - \mu_1}{\sigma^2} = 0\\
\frac{\partial l}{\partial \sigma^2 } = \sum_{i=1}^m -\frac{1}{2\sigma^2} + \frac{(x^{(i)} - \mu_{y^{(i)}})^2}{2\sigma^4} = 0
\end{aligned}
\]
And from these we get the estimations.
\end{answer}


} \fi
